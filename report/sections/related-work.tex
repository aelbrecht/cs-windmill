%%%%%%%%%%%%%%%%%%%%%%
\section{Related work}
%
%%%%%%%%%%%%%%%%%%%%%%

The most relevant project we found to what we set out to do is a project developed in Duke University that is planning to map the energy infrastructure of the world using satellite imagery \cite{duke}. They began with mapping all windmills in the USA and explained their approach \cite{duke_github} using a combination of machine learning and simulated imagery. They applied their models to colored satellite imagery and got satisfactory results. The project has at the point of writing this paper not expanded to any other areas or ocean based windmills. The project was also much larger in scale than what we set out to do, so we the idea of generating artificial imagery to train our model was not an option.\\

Another related project is written on Medium \cite{moraite_2019} which explained how it is relatively easy to identify ships using a CNN. This article has no sources for its data set due to link-rot; we assume it is related to another project using a different approach \cite{sagar_2019}. In both cases we can see that they used high resolution color satellite imagery. Both examples did not test on larger datasets, so we can assume that the model would probably not perform as well in a real world scenario. It gave us however a good example in how detecting ocean object can be done with even a simple neural network.\\
