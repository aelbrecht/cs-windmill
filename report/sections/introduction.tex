%%%%%%%%%%%%%%%%%%%%%%
\section{Introduction}
%
%%%%%%%%%%%%%%%%%%%%%%

The Sentinel-1 mission is the European Radar Observatory for the Copernicus joint initiative of the European Commission (EC) and the European Space Agency (ESA). Copernicus is a European initiative for the implementation of information services dealing with environment and security. It is based on observation data received from Earth Observation satellites and ground-based information.

The Sentinel-1 mission includes C-band imaging operating in four exclusive imaging modes with different resolution (down to 5 m) and coverage (up to 400 km). It provides dual polarisation capability, very short revisit times and rapid product delivery. For each observation, precise measurements of spacecraft position and attitude are available.

Synthetic Aperture Radar (SAR) has the advantage of operating at wavelengths not impeded by cloud cover or a lack of illumination and can acquire data over a site during day or night time under all weather conditions. Sentinel-1, with its C-SAR instrument, can offer reliable, repeated wide area monitoring.

The Sentinel-1 products are freely downloadable in GeoTIFF format.  GeoTIFF is a public domain metadata standard which allows georeferencing information to be embedded within a TIFF file. The potential additional information includes map projection, coordinate systems, ellipsoids, datums, and everything else necessary to establish the exact spatial reference for the file.

We aim to detect windmill parks in the ocean based on this satellite imagery.